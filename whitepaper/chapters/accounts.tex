\section{Accounts and Addresses}
\label{sec:accounts}

\nemquote{%
A beginning is the time for taking the most delicate care that the balances are correct.
}{Frank Herbert}

\codenamechapterfirstword uses elliptic curve cryptography to ensure confidentiality, authenticity and non-repudiability of all transactions.
An account is uniquely identified by an address, which is partially derived from a one way mutation of its signing public key.
Each account is linked to mutable state that is updated when transactions are accepted by the network.
This state is globally consistent and may contain zero or more public keys.

\subsection{Addresses}

\index{address!decoded}
A \emph{decoded address} is a 24 byte value composed of the following three parts:
\begin{itemize}
	\item{network byte}
	\item{160-bit hash of an account's signing public key}
	\item{3 byte checksum}
\end{itemize}

The checksum allows for quick recognition of mistyped addresses.
It is possible to send mosaics\footnote{A mosaic is a digital asset defined on \codename. Other technologies refer to mosaics as tokens.} to any valid address even if the address has not previously participated in any transaction.
If nobody owns the private key of the account to which the mosaics are sent, the mosaics are most likely lost forever.

\index{address!encoded}
An \emph{encoded address} is a Base32\footnote{\url{http://en.wikipedia.org/wiki/Base32}} encoding of a decoded address.
It is human readable and used in clients.

Notice that the Base32 encoding of binary data has an expansion factor of $\frac{8}{5}$, and a decoded address's size is not evenly divisible by five.
Accordingly, any Base32 representation of a decoded address will contain some extra information.
While this is not inherently problematic, when encoding a decoded address, the extra input byte is set to zero by convention for consistency across clients.
Additionally, the last byte of the resulting 40 character string is dropped.

\subsubsection{Address Derivation}
In order to convert a public key to an address, the following steps are performed:
\begin{enumerate}
	\item{Perform 256-bit SHA3 on the public key.}
	\item{Perform 160-bit RIPEMD\_160 of hash resulting from step 1.}
	\item{Prepend network version byte to RIPEMD\_160 hash.}
	\item{Perform 256-bit SHA3 on the result, take the first three bytes as a checksum.}
	\item{Concatenate output of step 3 and the checksum from step 4.}
	\item{Encode result using Base32.}
\end{enumerate}

\begin{figure}
	\nemcenterwithcaption{
		\begin{tikzpicture}[node distance=-\pgflinewidth]

	% First row, public key
	\node[draw,anchor=north west] (pubkeyLabel) at (0,0) {Public Key: };

	\node[crou,anchor=north west,minimum width=3cm] (pkx) at (3, 0) {X};
	\node[crou,minimum width=3cm, right=1cm of pkx] (pky) {Y};

	\node[fit=(pkx)(pky),crou] (xy) {};

	% join lines
	\node[crecb,below left=2cm and -10cm of pkx,text width=12cm] (bpk32) {32 bytes};

	% second row
	\draw[decorate,decoration={brace,raise=2pt}] (bpk32.north west) --
		node[above=4pt] (bpkLabel) {compressed-public-key}
		(bpk32.north east);

	\draw[dotted] (xy.west) -- (bpk32.north west);
	\draw[dotted] (xy.east) -- (bpk32.north east);

	% invisible
	\node[fit=(bpk32)(bpkLabel)] (bpk) {};


	% third row - ripeMD

	% it's easier to do it this way, as problems with paths show up when using nested tikzpicture
	\node[below left=1.5cm and -2cm of bpk,anchor=center] (ripeInner) { $\mathfunc{ripemd160}(\mathfunc{sha3\_256}($ };
	\node[right=of ripeInner, draw, dotted,rounded corners=0] (bpkInnerLabel) {compressed-public-key};
	\node[right=of bpkInnerLabel] (ripeInner2) { $)) $ };

	\node[fit=(ripeInner)(bpkInnerLabel)(ripeInner2), draw, dotted, rounded corners] (ripe) {};

	\draw[dotted] (bpk32.south west) -- (bpkInnerLabel.north west);
	\draw[dotted] (bpk32.south east) -- (bpkInnerLabel.north east);


	% Fourth row - network byte + ripemd output
	\node[crecb,below left=1cm and -1cm of ripe] (brmd1) {1};
	\node[crecb,right=of brmd1,text width=7.5cm] (brmd20) {20 bytes};

	\draw[dotted] (ripe.south west) -- (brmd20.north west);
	\draw[dotted] (ripe.south east) -- (brmd20.north east);

	% Fifth row - SHA3(version + ripe)
	\node[below right=2.5cm and -6cm of brmd20,anchor=center] (checkInner) { $\mathfunc{sha3\_256(}$ };
	\node[crecb,right=of checkInner] (bInnerRmd1) {1};
	\node[crecb,right=of bInnerRmd1,text width=7.5cm] (bInnerRmd20) {20 bytes};
	\node[right=of bInnerRmd20] (checkInner2) { $)) $ };

	\draw[dotted] (bInnerRmd1.north west) -- (brmd1.south west);
	\draw[dotted] (bInnerRmd20.north east) -- (brmd20.south east);

	\node[fit=(checkInner)(bInnerRmd1)(bInnerRmd20)(checkInner2), draw, dotted, rounded corners] (check) {};

	\draw[decorate,decoration={brace,mirror,raise=2pt},very thick] (check.south west) -- node[below=4pt] (check32) {32 bytes} (check.south east);


	% Sixth row - sha3 output
	\node[below=1.5cm of bInnerRmd20,text width=5.5cm] (check28) {\ldots 28 bytes};
	\node[crecb,left=of check28,text width=1.5cm] (check4) {4 bytes};
	\node[fit=(check4)(check28)] (checkGroup) {};
	\draw[decorate,decoration={brace,raise=2pt}] (checkGroup.north west) -- (checkGroup.north east);

	% Seventh row - binary address
	\node[crecb,below=2cm of check28,text width=1.5cm] (bAddr4) {4 bytes};
	\node[crecb,left=of bAddr4,text width=7.5cm] (bAddr20) {20 bytes};
	\node[crecb,left=of bAddr20] (bAddr1) {1};
	\node[fit=(bAddr1)(bAddr20)(bAddr4)] (binaryAddress) {};

	\draw[decorate,decoration={brace,raise=2pt},very thick] (bAddr1.north west) -- node[above=4pt] {binary address - 25 bytes} (bAddr4.north east);
	\draw[decorate,decoration={brace,mirror,raise=2pt},very thick] (bAddr1.south west) -- node[below=4pt] (binAddr) {} (bAddr4.south east);


	% final association
	\draw[dotted] (bAddr4.north west) -- (check4.south west);
	\draw[dotted] (bAddr4.north east) -- (check4.south east);

	\draw[dashed,thin] (bAddr1.north west) -- (brmd1.south west);
	\draw[dashed,thin] (bAddr20.north east) -- (brmd20.south east);

	\node[draw,below=2cm of binaryAddress,rounded corners,solid,inner sep=0.2cm] (nanemo) {
		MBEJBT-ECHBU7-3YM74A-YGDX6D-MANNITM-XDU3J-IMO2};
	\draw[solid,thin,->] (binaryAddress.south) -- node[right=1pt] {Base-32 encoding} (nanemo.north);

\end{tikzpicture}

	}{Address generation}
\end{figure}

\pagebreak

\subsubsection{Address Aliases}
\index{alias address}
An address can have one or more aliases assigned using an address alias transaction\nemtechdocsfootnote{concepts/namespace.html\#addressaliastransaction}.
All transactions accepting addresses support using either a public key derived address or an address alias.
In case of such transactions, the format of the address alias field is:
\begin{itemize}
	\item{network byte ORed with value 1}
	\item{8 byte namespace id that is an alias}
	\item{15 zero bytes}
\end{itemize}

\subsubsection{Intentional Address Collision}
It is possible that two different public signing keys will yield the same address.
If such an address contains any valuable assets \textbf{AND} has not been associated with a public key earlier (for example, by sending a transaction from the account), it would be possible for an attacker to withdraw funds from such an account.

In order for the attack to succeed, the attacker would need to find a private+public keypair such that the SHA3\_256 of the public key would \textbf{at the same time} be equal to the RIPEMD\_160 preimage of the 160-bit hash mentioned above.
Since SHA3\_256 offers 128 bits of security, it is mathematically improbable for a single SHA3\_256 collision to be found.
Due to similarities between \codenamespace addresses and Bitcoin addresses, the probability of causing a \codenamespace address collision is roughly the same as that of causing a Bitcoin address collision.

\subsection{Public Keys}
\label{sec:accounts:keys}

An account is associated with zero or more public keys. The supported types of keys are:
\begin{enumerate}
	\item{signing: \\
		ED25519 public key that is used for signing and verifying data.
		Any account can receive data, but only accounts with this public key can send data.
		This public key is the only one used as input into an account's address calculation.
	}
	\item{linked: \\
		This public key links a main account with a remote harvester.
		For a \field{Main} account, this public key specifies the remote harvester account that can sign blocks on its behalf.
		For a \field{Remote} account, this public key specifies the main account for which it can sign blocks.
		These links are bidirectional and always set in pairs.
	}
	\item{node: \\
		This public key is set on a \field{Main} account.
		It specifies the public key of the node that it can delegate harvest on.
		Importantly, this does not indicate that the remote is actively harvesting on the node, but only that it has the permission to do so.
		As long as either the harvesting account or the node owner is honest, the account will be restricted to delegate harvesting on a single node at a time.

		An honest harvester should only send its remote harvester private key to a single node at a time.
		Changing its remote will invalidate all previous remote harvesting permissions granted to all other nodes (and implies forward security of delegated keys).
		Any older remote harvester private keys will no longer be valid and unable to be used for harvesting blocks.
		An honest node owner should only ever remote harvest with a remote harvester private key that is \emph{currently} linked to its node public key
	}
	\item{VRF: \\
		ED25519 public key that is used for generating and verifying random values.
		This public key must be set on a \field{Main} account in order for the account to be eligible to harvest.
	}
	\item{voting: \\
		BLS public key that is used for signing and verifying finalization messages.
		All voting keys are temporary and must be registered with both a start and end epoch.
		This public key must be set on a \field{Main} account in order for the account to be eligible to vote.
		It is only valid when the current epoch is within its registered range.
		In order to allow a seamless switchover of keys, an account can have at most \nemsetting{network}{maxVotingKeysPerAccount} voting keys registered at once.
	}
\end{enumerate}
