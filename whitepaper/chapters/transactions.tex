\section{Transactions}
\label{sec:transactions}

\nemquote{
}{}

\nemchapterfirstletter{T}{ransactions} are instructions that modify the global chain state.
They are processed atomically and grouped into blocks.
If any part of a transaction fails processing, the global chain state is reset to the state prior to the transaction application attempt.

There are two fundamental types of transactions: basic transactions and aggregate transactions.
Basic transactions represent a single operation and require a single signature.
Aggregate transactions are containers of one or more transactions that may require multiple signatures.

Aggregate transactions allow basic transactions to be combined into potentially complex operations and executed atomically.
This increases developer flexibility relative to a system that only guarantees atomicity for individual operations while still constraining the global set of operations allowed to a finite set.
It does not require the introduction of a Turing-complete language and all of its inherent disadvantages.

\subsection{Basic Transaction}

A basic transaction is composed of both cryptographically verifiable and unverifiable data.
All verifiable data is contiguous and is signed by the transaction signer.
All unverifiable data is either ignored (e.g. padding bytes) or deterministically computable from verifiable data.
Each basic transaction requires verification of exactly one signature.

None of the unverifiable header fields need to be verified.
\texttt{Size} is the serialized size of the transaction and can always be derived from verifiable transaction data.
\texttt{Signature} is an output from signing and an input to verification.
\texttt{SignerPublicKey} is an input to both signing and verification.
In order for a transaction to pass signature verification, both \texttt{Signature} and \texttt{SignerPublicKey} must be matched with the verifiable data, which has a length relative to \texttt{Size}.
$$verify(T::Signature, T::SignerPublicKey, VerifiableDataBuffer(T))$$

\texttt{Reserved} bytes are used for padding transactions so that all integral fields and cryptographic primitives have natural alignment.
Since these bytes are meaningless, they can be stripped without invalidating any cryptographic guarantees.

Binary layouts for all transaction types are specified in catapult's open source schema language
\footnote{Schemas can be found at https://github.com/nemtech/catbuffer}.
Please refer to the published schemas for the most up to date specifications.

\begin{figure}[H]
	\nemcenterwithcaption{
		\begin{subfigure}{.5\textwidth}
			\nemmemorylayout{
				\begin{leftwordgroup}{\texttt{0x00}}
					\bitbox{4}{Size} \bitbox{4}{Reserved}
				\end{leftwordgroup} \\
				\nemmemorymultiwordbox{\texttt{0x08}}{Signature}{7} \\
				\nemmemorymultiwordbox{\texttt{0x48}}{SignerPublicKey}{3} \\
				\begin{leftwordgroup}{\texttt{0x68}}
					\bitbox{4}{Reserved} \bitbox[]{4}{}
				\end{leftwordgroup}
			}{Header (unsigned)}
		\end{subfigure}%
		\begin{subfigure}{.5\textwidth}
			\nemmemorylayout{
				\begin{leftwordgroup}{\texttt{0x6C}}
					\bitbox[br]{4}{} \bitbox{4}{V, N, T}
				\end{leftwordgroup} \\
				\begin{leftwordgroup}{\texttt{0x70}}
					\wordbox[tlr]{1}{MaxFee}
				\end{leftwordgroup} \\
				\begin{leftwordgroup}{\texttt{0x78}}
					\wordbox{1}{Deadline}
				\end{leftwordgroup} \\
				\nemmemorymultiwordboxskipped{\texttt{0x80}}{Payload}{} \\
			}{Verifiable data}
		\end{subfigure}
	}{Basic transaction binary layout}
\end{figure}

\subsection{Aggregate Transaction}

The layout of an aggregate transaction is more complex than that of a basic transaction, but there are some similarities.
An aggregate transaction shares the same unverifiable header as a basic transaction, and this data is processed in the same way.
Additionally, an aggregate transaction has a footer of unverifiable data followed by embedded transactions and cosignatures.

An aggregate transaction can always be submitted to the network with all requisite cosignatures.
In this case, it is said to be \emph{complete} and it is treated like any other transaction without any special processing.

API nodes can also accept \emph{bonded} aggregate transactions that have incomplete cosignatures.
The submitter must pay a bond that is returned if and only if all requisite cosignatures are collected before the transaction times out.
Assuming this bond is paid upfront, an API node will collect cosignatures associated with this transaction until it either has sufficient signatures or times out.

\texttt{TransactionsHash} is the most important field in an aggregate transaction.
It is the merkle root hash of the hashes of the embedded transactions stored within the aggregate.
In order to compute this field, a merkle tree is constructed by adding each embedded transaction hash in natural order.
The resulting root hash is assigned to this field.

None of the unverifiable footer fields need to be verified.
\texttt{PayloadSize} is a computed size field that must be correct in order to extract the exact same embedded transactions that were used to calculate \texttt{TransactionsHash}.
\texttt{Reserved} bytes, again, are used for padding and have no intrinsic meaning.

\begin{figure}[H]
	\nemcenterwithcaption{
		\begin{subfigure}{.33\textwidth}
			\nemmemorylayout{
				\begin{leftwordgroup}{\texttt{0x00}}
					\bitbox{4}{Size} \bitbox{4}{Reserved}
				\end{leftwordgroup} \\
				\nemmemorymultiwordbox{\texttt{0x08}}{Signature}{7} \\
				\nemmemorymultiwordbox{\texttt{0x48}}{SignerPublicKey}{3} \\
				\begin{leftwordgroup}{\texttt{0x68}}
					\bitbox{4}{Reserved} \bitbox[]{4}{}
				\end{leftwordgroup}
			}{Header (unsigned)}
		\end{subfigure}%
		\begin{subfigure}{.33\textwidth}
			\nemmemorylayout{
				\begin{leftwordgroup}{\texttt{0x6C}}
					\bitbox[br]{4}{} \bitbox{4}{V, N, T}
				\end{leftwordgroup} \\
				\begin{leftwordgroup}{\texttt{0x70}}
					\wordbox[tlr]{1}{MaxFee}
				\end{leftwordgroup} \\
				\begin{leftwordgroup}{\texttt{0x78}}
					\wordbox{1}{Deadline}
				\end{leftwordgroup} \\
				\begin{leftwordgroup}{\texttt{0x80}}
					\wordbox{1}{TransactionsHash}
				\end{leftwordgroup}
			}{Verifiable data}
		\end{subfigure}%
		\begin{subfigure}{.33\textwidth}
			\nemmemorylayout{
				\begin{leftwordgroup}{\texttt{0x88}}
					\bitbox{4}{\tiny PayloadSize} \bitbox{4}{Reserved}
				\end{leftwordgroup} \\
				\nemmemorymultiwordboxskipped{\texttt{0x90}}{Embedded}{Transactions} \\
				\nemmemorymultiwordboxskipped{}{Cosignatures}{}
			}{Unverifiable footer}
		\end{subfigure}
	}{Aggregate transaction header binary layout}
\end{figure}

\subsubsection{Embedded Transaction}

An embedded transaction is a transaction that is contained within an aggregate transaction.
Compared to a basic transaction, the header is smaller, but the transaction-specific data is the same.
\texttt{Signature} is removed because all signature information is contained in cosignatures.
\texttt{MaxFee} and \texttt{Deadline} are removed because they are specified by the parent aggregate transaction.

Client implementations can use the same code to construct the custom parts of either a basic or embedded transaction.
The only difference is in the creation and application of different headers.

Not all transactions are supported as embedded transactions.
For example, an aggregate transaction cannot be embedded within another aggregate transaction.

\begin{figure}[H]
	\nemcenterwithcaption{
		\begin{subfigure}{.5\textwidth}
			\nemmemorylayout{
				\begin{leftwordgroup}{\texttt{0x00}}
					\bitbox{4}{Size} \bitbox{4}{Reserved}
				\end{leftwordgroup} \\
				\nemmemorymultiwordbox{\texttt{0x08}}{SignerPublicKey}{3} \\
				\begin{leftwordgroup}{\texttt{0x28}}
					\bitbox{4}{Reserved} \bitbox[]{4}{}
				\end{leftwordgroup}
			}{Header (unsigned)}
		\end{subfigure}%
		\begin{subfigure}{.5\textwidth}
			\nemmemorylayout{
				\begin{leftwordgroup}{\texttt{0x2C}}
					\bitbox[br]{4}{} \bitbox{4}{V, N, T}
				\end{leftwordgroup} \\
				\nemmemorymultiwordboxskipped{\texttt{0x30}}{Payload}{}
			}{Verifiable data}
		\end{subfigure}
	}{Embedded transaction binary layout}
\end{figure}

\subsubsection{Cosignature}

A cosignature is a pair composed of a public key and its corresponding signature.
Zero or more cosignatures are appended at the end of an aggregate transaction.
Cosignatures are used to cryptographically verify an aggregate transaction involving multiple parties.

\begin{figure}[H]
	\nemcenter{
		\nemmemorylayout{
			\nemmemorymultiwordbox{\texttt{0x00}}{SignerPublicKey}{3} \\
			\nemmemorymultiwordbox{\texttt{0x20}}{Signature}{7}
		}{Cosignature binary layout}
	}{}
\end{figure}

In order for an aggregate transaction \texttt{A} to pass verification, it must pass basic transaction signature verification and have a cosignature for every embedded transaction signer
\footnote{In the case of multisignature accounts, there must be enough cosignatures to satisfy the multisignature account constraints.}.

Like any transaction, an aggregate transaction, must pass basic transaction signature verification.
$$verify(A::Signature, A::SignerPublicKey, VerifiableDataBuffer(A))$$

Additionally, all cosignatures must pass signature verification.
Notice that cosigners sign the hash of an aggregate transaction data, not the data directly.
$$\sum_{0\leq i \leq N_C} verify(C::Signature, C::SignerPublicKey, H(VerifiableDataBuffer(A)))$$

Finally, there must be a cosignature that corresponds to and satisfies each embedded transaction signer.

\subsubsection{Extended Layout}

The aggregate transaction layout described earlier was correct with one simplification.
All embedded transactions are padded so that they end on 8-byte boundaries.
This ensures that all embedded transactions and cosignatures begin on 8-byte boundaries as well.
The padding bytes are never signed nor included in any hashes.

\begin{figure}[H]
	\nemcenter{
		\nemmemorylayout{
			\begin{leftwordgroup}{}
				\bitbox{4}{\tiny PayloadSize} \bitbox{4}{Reserved}
			\end{leftwordgroup} \\
			\nemmemorymultiwordboxskipped{}{Embedded}{Transaction 1} \\
			\begin{leftwordgroup}{}
				\wordbox{1}{\emph{optional} padding}
			\end{leftwordgroup} \\
			\nemmemorymultiwordboxskipped{}{Embedded}{Transaction 2} \\
			\begin{leftwordgroup}{}
				\wordbox{1}{\emph{optional} padding}
			\end{leftwordgroup} \\
			\nemmemorymultiwordboxskipped{}{Cosignatures}{}
		}{Aggregate transaction footer with padding}
	}
\end{figure}
